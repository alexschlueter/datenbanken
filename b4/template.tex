%%%%%%%%%%%%%%%%%%%%%%%%%%%%%%%%%%%%%%%%%%%%%%%%%%%%%%%%%%%%%%%%%%%%%%%%%%%%%%%%%%%%%%%%%
%             template.tex                                                              %
%                                                                                       %
%            Author: Sergej Lewin 10/2008                                               %
%                                                                                       %    
% !!!Man braucht noch die Datei Ueb.sty (im gleichen Ordner wie die Hauptdatei)!!!      %
%%%%%%%%%%%%%%%%%%%%%%%%%%%%%%%%%%%%%%%%%%%%%%%%%%%%%%%%%%%%%%%%%%%%%%%%%%%%%%%%%%%%%%%%%
\documentclass[a4paper,11pt]{article}             % bestimmt das Aussehen eines Dokuments
\usepackage{Ueb}                                  % vordefinierte Makros

%!!!!anpassen an das Betriebssystem!!!, um Umlaute zu verwenden
\usepackage[utf8]{inputenc}                      %Linux
%\usepackage[latin1]{inputenc}                    %Windows
%\usepackage[applemac]{inputenc}                  %Mac



%Namen und Matrikelnummern anpassen
%\zweinamen{Name1}{Matrikelnummer1}{Name2}{Matrikelnummer2} %2er Gruppen
\dreinamen{Tanja Wilke}{Jan Rathner}{Alexander Schlüter} %3er Gruppe

%Briefkastennummer anpassen. z. B. \briefkasten{104}
\briefkasten{83}

%Termin der Uebungsgruppe und Raum anpassen z. B. \termin{Mo. 12-14 , SR2}
\termin{Do. 8-10, SR1A}

%Blattnummer anpassen z. B. \blatt{5}
\blatt{4}

\begin{document}
%Hier kommt der Text des Dokuments......
\setcounter{excnt}{15}
\begin{ex}
  Die Anfrage liefert nicht das gewünschte Ergebnis, da sowohl das
  Relationsschema \textbf{Benutzer} als auch das Relationsschema
  \textbf{Computer} ein Attribut mit Namen \emph{Anmerkungen} hat.
  Beim Equijoin werden die Tupel anhand dieses Attributes verglichen und da die
  Anmerkungen überall unterschiedlich sind, ist das Ergebnis des Equijoins eine
  leere Relation. Stattdessen sollte man mindestens eins der Anmerkungsattribute vor dem
  Equijoin umbenennen:
  \begin{equation*}
   \Pi_{\text{PersonalNr,Name,Aufträge}}(\rho_{(\text{PersonalNr,Name,B.Anmerkungen})}(\text{Benutzer})\bowtie\text{gehört}\bowtie\text{Computer}) 
  \end{equation*}
  Man könnte das Anmerkungsattribut auch wegfallen lassen, da es hinterher von
  der Projektion sowieso gelöscht wird:
  \begin{equation*}
   \Pi_{\text{PersonalNr,Name,Aufträge}}(\Pi_{\text{PersonalNr,Name}}(\text{Benutzer})\bowtie\text{gehört}\bowtie\text{Computer}) 
  \end{equation*}
\end{ex}
\end{document}
