%%%%%%%%%%%%%%%%%%%%%%%%%%%%%%%%%%%%%%%%%%%%%%%%%%%%%%%%%%%%%%%%%%%%%%%%%%%%%%%%%%%%%%%%%
%             template.tex                                                              %
%                                                                                       %
%            Author: Sergej Lewin 10/2008                                               %
%                                                                                       %    
% !!!Man braucht noch die Datei Ueb.sty (im gleichen Ordner wie die Hauptdatei)!!!      %
%%%%%%%%%%%%%%%%%%%%%%%%%%%%%%%%%%%%%%%%%%%%%%%%%%%%%%%%%%%%%%%%%%%%%%%%%%%%%%%%%%%%%%%%%
\documentclass[a4paper,11pt]{article}             % bestimmt das Aussehen eines Dokuments
\usepackage{Ueb}                                  % vordefinierte Makros

%!!!!anpassen an das Betriebssystem!!!, um Umlaute zu verwenden
\usepackage[utf8]{inputenc}                      %Linux
%\usepackage[latin1]{inputenc}                    %Windows
%\usepackage[applemac]{inputenc}                  %Mac



%Namen und Matrikelnummern anpassen
%\zweinamen{Name1}{Matrikelnummer1}{Name2}{Matrikelnummer2} %2er Gruppen
\dreinamen{Tanja Wilke}{Jan Rathner}{Alexander Schlüter} %3er Gruppe

%Briefkastennummer anpassen. z. B. \briefkasten{104}
\briefkasten{83}

%Termin der Uebungsgruppe und Raum anpassen z. B. \termin{Mo. 12-14 , SR2}
\termin{Do. 8-10, SR1A}

%Blattnummer anpassen z. B. \blatt{5}
\blatt{2}

\begin{document}
%Hier kommt der Text des Dokuments......
\setcounter{excnt}{6}
\begin{ex}
\begin{exlist}
  \leavevmode
  \item 
Die Auflösung ist nicht äquivalent, da das rechte Modell nicht sicherstellt,
dass zu jeder Entität $e$ vom Typ $E$ jeweils genau eine Entität $a$ vom Typ $A$,
eine Entität $b$ vom Typ $B$ und eine Entität $c$ vom Typ $C$ existieren, 
sodass $(a,e)\in R_A$, $(b,e)\in R_B$ und $(c,e)\in R_C$ gilt.

Zum Beispiel erlaubt das rechte Modell eine Instanz mit den Entitätsmengen
(benutze selbe Symbole wie für Entitätstypen):
\begin{align*}
  A&=\set{a},\quad B=C=\varnothing,\quad E=\set{e} \\
  R_A&=\set{(a,e)},\quad R_B=R_C=\varnothing
\end{align*}
Dies lässt sich nicht in das linke Modell übersetzen.

\item
Es muss vollständige Teilnahme von $E$ an $R_A$, $R_B$ und $R_C$ gelten, d.h.
jede Entität vom Typ $E$ muss jeweils in mindestens einer Beziehung vorkommen.
Außerdem darf jede Entität vom Typ $E$ nur \emph{genau} einmal in den
Beziehungen vorkommen. D.h. die Kardinalität der Beziehungstypen $R_A$, $R_B$,
$R_C$ muss so angepasst werden, dass die Verbindungen zum Entitätstyp $E$ den
(min, max) Bezeichner $(1,1)$ bekommen. 
\vspace{2cm}
\item

  \begin{center}
\usetikzlibrary{positioning}
\usetikzlibrary{shadows}

\tikzstyle{every entity} = [top color=white, bottom color=blue!30, 
                            draw=blue!50!black!100, drop shadow]
\tikzstyle{every weak entity} = [drop shadow={shadow xshift=.7ex, 
                                 shadow yshift=-.7ex}]
\tikzstyle{every attribute} = [top color=white, bottom color=yellow!20, 
                               draw=yellow, node distance=1cm, drop shadow]
\tikzstyle{every relationship} = [top color=white, bottom color=red!20, 
                                  draw=red!50!black!100, drop shadow]
\tikzstyle{every isa} = [top color=white, bottom color=green!20, 
                         draw=green!50!black!100, drop shadow]

\resizebox{0.8\textwidth}{!}{
\begin{tikzpicture}[node distance=1.5cm, every edge/.style={link},baseline]

  \node[entity] (A) {A};
  \node[relationship] (RA) [below=1cm of A] {$R_A$} edge node [right] {(0,N)} (A);
  \node[entity] (E) [below=1cm of RA] {E} edge [total] node [right] {(1,1)} (RA);
  \node[relationship] (RB) [left=1cm of E] {$R_B$} edge [total] node [above] {(1,1)} (E);
  \node[entity] (B) [left=1cm of RB] {B} edge node [above] {(0,N)} (RB);
  \node[relationship] (RC) [right=1cm of E] {$R_C$} edge [total] node [above] {(1,1)} (E);
  \node[entity] (C) [right=1cm of RC] {C} edge node [above] {(0,N)} (RC);

\end{tikzpicture}
}
\end{center}
\vspace{2cm}
\item Diese Alternative ist auch nicht äquivalent zum ersten Modell, da es nicht
  möglich ist, eine Beziehung zwischen \emph{drei} Entitäten darzustellen. Sind
  zum Beispiel $a\in A$, $b\in B$ und $c_1, c_2\in C$ und gibt es im
  ursprünglichen Modell die Beziehungen $(a, b, c_1)\in R$ und $(a, b, c_2)\in
  R$, so kann dies nicht ohne Informationsverlust in das alternative Modell
  übertragen werden.
\end{exlist}
\end{ex}
\end{document}
