%%%%%%%%%%%%%%%%%%%%%%%%%%%%%%%%%%%%%%%%%%%%%%%%%%%%%%%%%%%%%%%%%%%%%%%%%%%%%%%%%%%%%%%%%
%             template.tex                                                              %
%                                                                                       %
%            Author: Sergej Lewin 10/2008                                               %
%                                                                                       %    
% !!!Man braucht noch die Datei Ueb.sty (im gleichen Ordner wie die Hauptdatei)!!!      %
%%%%%%%%%%%%%%%%%%%%%%%%%%%%%%%%%%%%%%%%%%%%%%%%%%%%%%%%%%%%%%%%%%%%%%%%%%%%%%%%%%%%%%%%%
\documentclass[a4paper,11pt]{article}             % bestimmt das Aussehen eines Dokuments
\usepackage{Ueb}                                  % vordefinierte Makros

%!!!!anpassen an das Betriebssystem!!!, um Umlaute zu verwenden
\usepackage[utf8]{inputenc}                      %Linux
%\usepackage[latin1]{inputenc}                    %Windows
%\usepackage[applemac]{inputenc}                  %Mac



%Namen und Matrikelnummern anpassen
%\zweinamen{Name1}{Matrikelnummer1}{Name2}{Matrikelnummer2} %2er Gruppen
\dreinamen{Tanja Wilke}{Jan Rathner}{Alexander Schlüter} %3er Gruppe

%Briefkastennummer anpassen. z. B. \briefkasten{104}
\briefkasten{83}

%Termin der Uebungsgruppe und Raum anpassen z. B. \termin{Mo. 12-14 , SR2}
\termin{Do. 8-10, SR1A}

%Blattnummer anpassen z. B. \blatt{5}
\blatt{3}

\begin{document}
%Hier kommt der Text des Dokuments......
\setcounter{excnt}{11}
\begin{ex}
\begin{exlist}
  \leavevmode
  \item
  \textbf{Entwurf 1}: Nein, es muss für jede Betriebssysteminstanz mit Typ Linux eine Beziehung für das Ticket erstellt werden. \\
\textbf{Entwurf 2}:  Ja, denn es genügt, eine einzige Beziehung \emph{betrifft\_b} zwischen dem Ticket und der Betriebssystemtypentität Linux zu erstellen. \\
\textbf{Entwurf 3}: Ja, wie bei Entwurf 2 genügt es eine Beziehung hinzuzufügen. 
\item
\textbf{Entwurf 1}: Ja, es können beliebig viele Entitäten
\textit{Betriebsysteminstanz} mit beliebigem Typ erstellt werden. \\
\textbf{Entwurf 2}: Ja, auch hier könne beliebig Betriebsysteminstanzen
hinzugefügt werden. \\
\textbf{Entwurf 3}: Nein, hier gibt es mehrere Probleme. Die Spezialisierung von
Betriebssystemtyp zu Betriebssysteminstanz ist vollständig, also muss es für
jeden Typen eine Instanz geben. Außerdem muss das \emph{Typ}-Attribut eindeutig
sein, da es als Primärschlüssel gekennzeichnet ist. Da durch die Spezialisierung
jede Entität \emph{Betriebssysteminstanz} auch eine Entität \emph{Betriebssystemtyp}
ist, darf es von jedem Typen nur eine Instanz geben (wegen der Vollständigkeit
also genau eine).
\end{exlist}
\end{ex}
\end{document}
