%%%%%%%%%%%%%%%%%%%%%%%%%%%%%%%%%%%%%%%%%%%%%%%%%%%%%%%%%%%%%%%%%%%%%%%%%%%%%%%%%%%%%%%%%
%             template.tex                                                              %
%                                                                                       %
%            Author: Sergej Lewin 10/2008                                               %
%                                                                                       %    
% !!!Man braucht noch die Datei Ueb.sty (im gleichen Ordner wie die Hauptdatei)!!!      %
%%%%%%%%%%%%%%%%%%%%%%%%%%%%%%%%%%%%%%%%%%%%%%%%%%%%%%%%%%%%%%%%%%%%%%%%%%%%%%%%%%%%%%%%%
\documentclass[a4paper,11pt]{article}             % bestimmt das Aussehen eines Dokuments
\usepackage{Ueb}                                  % vordefinierte Makros

%!!!!anpassen an das Betriebssystem!!!, um Umlaute zu verwenden
\usepackage[utf8]{inputenc}                      %Linux
%\usepackage[latin1]{inputenc}                    %Windows
%\usepackage[applemac]{inputenc}                  %Mac



%Namen und Matrikelnummern anpassen
%\zweinamen{Name1}{Matrikelnummer1}{Name2}{Matrikelnummer2} %2er Gruppen
\dreinamen{Tanja Wilke}{Jan Rathner}{Alexander Schlüter} %3er Gruppe

%Briefkastennummer anpassen. z. B. \briefkasten{104}
\briefkasten{83}

%Termin der Uebungsgruppe und Raum anpassen z. B. \termin{Mo. 12-14 , SR2}
\termin{Do. 8-10, SR1A}

%Blattnummer anpassen z. B. \blatt{5}
\blatt{5}

\begin{document}
%Hier kommt der Text des Dokuments......
\setcounter{excnt}{17}
\begin{ex}
  \begin{exlist}
    \leavevmode
  \item Wird zuerst das kartesische Produkt $r\times s$ gebildet, so werden
    sofort $n\cdot m$ Tupel erstellt, wenn $n$ die Anzahl der Tupel in $r$ und
    $m$ die Anzahl der Tupel in $s$ ist. Der Equijoin ist wahrscheinlich in den
    meisten Implementationen schneller, weil bei der Iteration über die Tupel
    sofort die Attribute im Schnitt verglichen werden können und dadurch eine
    Explosion der Tupelzahl verhindert werden kann.
   
  \item  
    \begin{exlist}
    \item Damit die Ausdrücke sinnvoll sind, dürfen in $P$ und $Q$ nur die
      Attribute von $R$ vorkommen. Dann gilt die Äquivalenz ohne
      Einschränkungen. Möglicherweise ist die erste Variante schneller, da nicht
      zweimal über die Tupel iteriert werden muss.
    \item In $X$ dürfen nur Attribute von $R$ vorkommen und in $P$ nur solche,
      die in $X$ vorkommen. Wenn $P$ nur wenige Tupel selektiert, dann ist die
      zweite Variante schneller, da die Tupelanzahl schnell reduziert wird. Gibt
      es viele Tupel, die in den Attributen in $X$ übereinstimmen, dann wird die
      Tupelanzahl durch die Projektion stark reduziert und die erste Variante
      ist vielleicht schneller.
    \item Dies ist nur sinnvoll, wenn in $X$ nur Attribute vorkommen, die auch
      in $Y$ vorkommen. Dann sind die Ausdrücke äquivalent. Die zweite Variante
      ist möglicherweise schneller, da nur einmal über die Tupel iteriert werden muss.
    \item In $P$ dürfen nur Attribute aus $R$ vorkommen. Die zweite Variante ist
      vielleicht schneller, da die Tupelanzahl stark reduziert werden kann, wenn
      $P$ nur wenige Tupel selektiert. Der Equijoin ist mit $n\cdot m$ nötigen
      Vergleichen relativ langsam, also lohnt sich eine vorherige Reduzierung
      der Tupelanzahl.
    \end{exlist}
  \item Der folgende Ausdruck ist effizienter:
    \begin{equation*}
     \Pi_{A,C}(\sigma_{A>5}(r))\bowtie\Pi_{C,E}(s)
    \end{equation*}
  \end{exlist}
\end{ex}
\begin{ex}
\begin{exlist}
\leavevmode 
\item 
  \begin{exlist}
  \item Die Namen aller Karteninhaber, die in Münster wohnen und Studenten sind
    oder die in Kattenvenne wohnen und kein Student sind. 
  \item Die Bezeichner aller Gerichte, die maximalen Preis unter den Gerichten haben.
  \item Die Namen aller Karteninhaber, mit deren Karte nicht am 01.02.2016 in
    der Mensa am Ring ein Gyrosteller gekauft wurde.
  \end{exlist}
\item 
  \begin{exlist}
  \item $\begin{aligned}[t]\{(k\text{.Inhabername}, m\text{.Guthaben})\mid
      k\in\text{karteninhaber}&\land m\in\text{mensakarte}\\
&\land k.\text{KartenID}=m.\text{KartenID}\}\end{aligned}$
\item $\begin{aligned}[t]\{(k\text{.Inhabername}, k\text{.Wohnort})\mid
    k\in\text{karteninhaber}&\land\exists
    l\in\text{kauft}(l.\text{Bezeichner}=\text{'Pommes'}\\
    &\land\exists b\in\text{bezahlung}(b.\text{Nummer}=l.\text{Nummer}\\
    &\land b.\text{KartenID}=k.\text{KartenID}))\}\end{aligned}$
  \item $\begin{aligned}[t]\{(m\text{.Mensaname})\mid
      m\in\text{mensa}&\land \forall l\in\text{mensa}(m.\text{Sitzplätze}\leq l.\text{Sitzplätze})\}\end{aligned}$
  \item $\begin{aligned}[t]\{(m\text{.Mensaname})\mid
      m\in\text{mensa}&\land \nexists
      b\in\text{bezahlung}(b.\text{MensaName}=m.\text{Mensaname}\\
&\land\exists
      k\in\text{karteninhaber}(k.\text{KartenID}=b.\text{KartenID}\\
&\land
      k.\text{Inhabername}=\text{'Aaron Atom'}))\}\end{aligned}$
  \item $\begin{aligned}[t]\{(m\text{.Mensaname})\mid
      m\in\text{mensa}&\land \forall a\in\text{bietetan}(a.\text{Mensaname}\neq
      m.\text{Mensaname}\\
&\lor a.\text{Datum}\neq\text{'03.04.2016'}\\
&\lor\exists
      k1\in\text{kauft}(k1.\text{Bezeichner}=a.\text{Bezeichner}\\
&\land\exists
      k2\in\text{kauft}(k2.\text{Bezeichner}=a.\text{Bezeichner}\\
&\land
      k2.\text{Nummer}\neq k1.\text{Nummer})))\}\end{aligned}$
  \end{exlist}
\end{exlist}
\end{ex}
\end{document}
